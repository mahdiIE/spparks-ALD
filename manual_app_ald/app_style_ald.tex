\documentclass[a4paper,12pt]{report}
\usepackage[version=3]{mhchem} % Formula subscripts using \ce{}
\newcommand{\RNum}[1]{\uppercase\expandafter{\romannumeral #1\relax}}
\usepackage{graphicx}
\usepackage{caption}
\usepackage{subcaption}
\usepackage{amssymb}
\usepackage{cite}
\setlength{\parindent}{0cm}


% Title Page
\title{ALD application}
\author{Mahdi Shirazi, Simon Elliott}


\begin{document}
\maketitle

\textbf{Syntax:}
\newline
  \emph{app$_-$style ald}

  \begin{itemize}

  \item \emph{ald=style name of this application}

  \end{itemize}

\textbf{Examples:}
\newline
  \emph{app$_-$style ald}
\newline

\textbf{Description:}
\newline
This application simulates the chemical reaction of atomic layer deposition (ALD) on a crystalline hafnium oxide lattice.
This application reads in all lattice sites via the read$_-$sites command.

This application stores 2 integers per lattice site. The first integer (i1) is the element on the site:

\begin{itemize}

  \item \emph{element=1=O}
  \item \emph{element=2=OH}
  \item \emph{element=3=HfX4O (adsorbed precursor and X shows amide group)}
  \item \emph{...}

\end{itemize}

The second integer (i2) is the 'coordination' of the site.

The 3$_-$fold lattice should be created using the read$_-$sites, which gives details of its geometry, initial value, and neighbor connectivity.
The 3$_-$fold lattice is initialized by low coordinated \ce{O} and \ce{OH} as substrate.
The remainder are vacant.

The event command is used to define what kinds of diffusion, dissociation, and densification occur in the model.
The relative probability of each event is computed as a function of the rate, coordination number, and time specified in the event command.
The details explained in the event command.
Temperature is specified by the temperature command. 

This application includes the interaction between the remaining precursor (steric effect) and influence of remaining fragments on adsorption sties (blocking).
This is implemented using the neighbor list (see read$_-$sites).

This application can be evolved only by a kinetic Monte Carlo (KMC). You must thus define a KMC solver to be used with the application via the solve$_-$style command.
\newline

\textbf{Restrictions:}
\newline

This application can only be run in serial simulations, not parallel. 
%In serial, it can not be used with multiple sectors, only one sector.
\newline

\textbf{event command:}
\newline

\emph{event type old1 new1 old2 new2 A n E coord pressureOn expression}

\begin{itemize}

  \item \emph{type=1,2,3}
  \item \emph{old1,old2=chemical species before reaction occurring}
  \item \emph{new1,new2=chemical species after reaction occurring}
  \item \emph{A= pre-factor or rate of adsorption}
  \item \emph{n=exponential factor in Arrhenius equation}
  \item \emph{E=activation energy}
  \item \emph{coord=c.n. of the first lattice site} 
  \item \emph{pressureOn=0,1,2}
  \item \emph{expression=details of reaction}

\end{itemize}

\textbf{Examples:}
\newline

{\footnotesize{\emph{event   1       O               HfX4O            44879.2084             0       0.00    1       1       HfX4(g)+O(s)$\to$HfX4...O(s)
\newline
event   1       HfX4O           O               1.042296E13             0       1.00    2       0       HfX4(g)+O(s)$\to$HfX4...O(s)
\newline
event   1       HfH2X2          HfHX            1.042296E13             0       0.80    4       0       HfH2X2$\to$HfHX
\newline
event   1       HfH2X2          HfHX            1.042296E13             0       0.30    5       0       HfH2X2$\to$HfHX
\newline
event   2       O               OH              OH              O               1.042296E13             0       0.46    1       0       O$\to$OH
\newline
event   2       O               OH              OH              O               1.042296E13             0       0.75    2       0       O$\to$OH
\newline
event   2       HfX4O           HfX4OH          OH              O               1.042296E13             0       0.75    2       0       HfX4...O+OH$\to$HfX4...OH+O
\newline
event   2       HfX4OH          HfX4O           O               OH              1.042296E13             0       0.75    2       0               same
\newline
event   3       HfX2O           O               VAC             HfX2            1.042296E13             0       0.20    0       0       HfX2...O+VAC$\to$O+HfX2
\newline
event   3       HfX2OH          OH              VAC             HfX2            1.042296E13             0       0.20    0       0       HfX2...OH+VAC$\to$HfX2+OH
\newline
event   3       OH2HfX          HfX             VAC             OH2             1.042296E13             0       0.30    4       0       OH2HfX+VAC$\to$HfX+OH
\newline
event   3       OH2HfX          HfX             VAC             OH2             1.042296E13             0       0.30    5       0       OH2HfX+VAC$\to$HfX+OH
\newline}}}


\textbf{Description:}
\newline

This command defines an event for the "app$_-$style ald" application. 
It can be an event involving one or two lattice sites, as specified by \emph{type}. 
The \emph{type} 1 only changes the species of a site. 
The \emph{type} 2 and 3 change the species of a site and its second and first neighbor site respectively. 
App$_-$style ald operates on a 3-fold lattice which contains monoclinic hafnium oxide sites.
The chemical species in a site and its neighboring site specify the event. 
If \emph{type} = 1, then only \emph{old1} and \emph{new1} are specified. 
If \emph{type} = 2 or 3, then \emph{old1, new1, old2,} and \emph{new2} are specified.
The chemical specify what atoms or fragment must be on those sites in order for the event to potentially take place. 
The possible atoms and fragment are specified in in.ald file. 
\newline

For the adsorption events, \emph{A} shows the rate given by Maxwell-Boltzmann statistics,
while For the other events \emph{A} shows the pre-factor.
\newline

\emph{n} corresponds to 0 the original Arrhenius equation below.
\newline

\emph{E} shows activation energy and the rate setting determines the relative rate at which the event will occur. 
\newline

\emph{coord} shows c.n. of species in the first site.
E.g. in the first example above, an adsorption of precursor must be on an oxygen species with c.n.=1 for the event to be possible.
\newline

In ALD simulation, the respective adsorption reactions are turned on and off as simulation time advances. 
Adsorption reactions of the two precursors occur alternately as time progresses. 
\emph{pressureOn} = 0 shows that the reaction is possible all simulation time.
Similarly, 1 and 2 show metal pulse and oxygen pulse respectively.
This option also used to filter out wasteful events.
\newline

\emph{rate = A(T/T$_0$)$^n$exp(-E/kT)}
\newline

Where T is the temperature you have specified.
In this case the rate setting should be in the energy units defined by the application's Hamiltonian and should be
consistent with the units used in the temperature command.
\newline

\textbf{Pulse and purge time command:}
\newline

\textbf{Syntax:}
\newline
  \emph{pulse$_-$time T1 T3}
\newline
  \emph{purge$_-$time T2 T4}

  \begin{itemize}

  \item \emph{T1=time for the metal pulse}
  \item \emph{T2=time for the first purge}
  \item \emph{T3=time for the oxygen pulse}
  \item \emph{T4=time for the second purge}

  \end{itemize}

\textbf{Example:}

\emph{pulse$_-$time              0.0001  0.0001 
\newline
purge$_-$time              0.0001  0.0001} 
\newline

\textbf{Description:}
\newline

These commands set time for pulse and purge of a cycle in ald application.
In the ALD process, gaseous precursors are admitted to the reactor in alternate
pulses separated by periods of purging. To implement this in KMC, the respective
adsorption reactions are turned on and off as simulation time advances. 
Adsorption reactions of the two precursors occur alternately as time progresses. 
During the purge, no adsorption reaction is allowed.

\textbf{Diag$_-$style ald command:}
\newline

\textbf{Syntax:}
\newline
\emph{diag$_-$style ald keyword value keyword value ...}

  \begin{itemize}

  \item \emph{ald=style name of this diagnostic}
  \item \emph{zero or more keyword/value pairs may be appended}
  \item \emph{see the diag$_-$style command for additional keyword/value pairs that can be appended to a diagnostic command
             and which must appear before these keywords}
  \item \emph{keyword=list}
  \item \emph{list values = events or O or OH or HfX4O or ... or s1 or d1 or v1 or ...
             \newline
             O, OH, HfX4O = counts of how many lattice sites of this type exist
             \newline
             events = total \# of events for all sites
             \newline
             sN,dN,vN = cumulative \# of events for this reaction that have occurred.}

  \end{itemize}

\textbf{Examples:}
\newline
\emph{diag$_-$style ald keyword yes list events  O OH Hf HfX4O HfHX4O HfH2X4O HfH3X4O HfH4X4O s78 v18}

\textbf{Description:}
\newline

The ald diagnostic prints out statistics about the system being modeled by app$_-$style ald.
The values will be printed as part of stats output and it can be in any order.

The O, OH, HfX4O and ... values will print counts of the number of chemical species of sites.
The events value will print the total \# of possible events that can occur as defined by the event command,
given the current state of the lattice, summed over all sites.

The sN, dN, and vN values refer to a specific events that have occurred, as defined by the event command.
The letter 's' means reactions involving a single site, 'd' means reactions involving a site and its second neighbor, 
and 'v' means reactions involving a site and its first neighbor.
The N refers to which reaction (from 1 to the number of the type of reaction).
For instance, 'v18' means the 18$^{th}$ of type III reaction defined in your input script.

\textbf{Restrictions:}
\newline

This command can only be used as part of the app$_-$style ald application.

%\textbf{Related commands:}
%\newline
%
%diag$_-$style, stats

\bibliography{citations}
\bibliographystyle{plain}

\end{document}          
